% main.tex – bereinigte Version
\documentclass[12pt,a4paper]{report}

% --- Eingabe & Schriftcodierung -------------------------------------------------
\usepackage[utf8]{inputenc}
\usepackage[T1]{fontenc}
\usepackage[german]{babel}

% --- Typografie & Layout --------------------------------------------------------
\usepackage{lmodern}
\usepackage{geometry}
\geometry{margin=2.5cm}
\usepackage{microtype}
\usepackage{parskip}
\usepackage{amsmath,amssymb}
\usepackage{enumitem}

% --- Links & PDF‑Metadaten ------------------------------------------------------
\usepackage{hyperref}
\hypersetup{
  colorlinks   = true,
  linkcolor    = blue,
  urlcolor     = blue,
  pdftitle     = {Überblick: Algorithmik und Theoretische Informatik},
  pdfauthor    = {Timon Bambynek, Max Mustermann, Maria Musterfrau}
}

% Doppelte „page.1“‑Zielmarke vermeiden: erst deaktivieren, nach Titel wieder aktiv.
\hypersetup{pageanchor=false}

% --- Kopf- / Fußzeilen ----------------------------------------------------------
\usepackage{fancyhdr}
\pagestyle{fancy}
\fancyhead{}
\fancyhead[L]{Alg. \& Theo. Informatik}
\fancyhead[R]{\nouppercase{\rightmark}}
\fancyfoot{}
\fancyfoot[C]{\thepage}
% Warnung von fancyhdr beseitigen
\setlength{\headheight}{14pt}
\addtolength{\topmargin}{-2pt}

% Automaten Design:
\usepackage{tikz}
\usetikzlibrary{arrows.meta,automata,positioning}
\tikzset{>=Stealth}

% (Pseudo) Code
\usepackage{algorithm}
\usepackage{algpseudocode}

% --- Titeldaten -----------------------------------------------------------------
\title{Überblick: Algorithmik und Theoretische Informatik\\[0.5em]\large Lern- und Übungsskript}
\author{Timon Bambynek \and Max Mustermann \and Maria Musterfrau}
\date{\today}

% ===============================================================================
\begin{document}
\hypersetup{pageanchor=true}% jetzt wieder Zielmarken erzeugen

% --- Titelblatt -----------------------------------------------------------------
\begin{titlepage}
  \centering
  {\scshape\LARGE FH Aachen (AMI im Rahmen der Matse Ausbildung)\par}
  \vspace{1.5cm}
  {\huge\bfseries Überblick: Algorithmik und Theoretische Informatik\par}
  \vspace{0.5cm}
  {\Large Lern- und Übungsskript\par}
  \vspace{1.5cm}
  {\Large Autoren:\par}
  \vspace{0.3cm}
  {\large Timon Bambynek\par}
  {\large Max Mustermann\par}
  {\large Maria Musterfrau\par}
  \vfill
  {\large \today\par}
\end{titlepage}

\pagenumbering{Roman}
\tableofcontents
\newpage
\pagenumbering{arabic}

% --- Kapitel --------------------------------------------------------------------
% chapters/automatentheorie.tex – vollständige, kompilierbare Version
% Benötigt im Hauptdokument:
%   \usepackage{tikz}
%   \usetikzlibrary{automata,positioning}
%   \tikzset{>=stealth'}

%========================================================
\chapter{Automatentheorie}

%--------------------------------------------------------
\section{Induktive Definitionen}
Induktive Definitionen gestatten es, \emph{unendliche} Mengen mithilfe endlich vieler Regeln festzulegen.\\[4pt]
\textbf{Schema}
\begin{itemize}
  \item \textit{Basis:} Festlegung mindestens eines trivialen Elements, z.\,B. $\varepsilon$ als leeres Wort.
  \item \textit{Induktionsschritt:} Beschreibung, wie aus bereits konstruierten Elementen neue entstehen, z.\,B.
        \[ w\in\Sigma^{\ast},\; a\in\Sigma \;\Longrightarrow\; w a\in\Sigma^{\ast}. \]
\end{itemize}
So erhält man z.\,B. die vollständige Wortmenge $\Sigma^{\ast}$ eines endlichen Alphabetes.  
In Beweisen arbeitet man \emph{vorwärts} (Aufbau) oder \emph{rückwärts} (Zerlegung in Basis + Schritt).

%--------------------------------------------------------
\section{Codierung von Daten: Zahlen, Folgen, Matrizen}
\begin{description}
  \item[Zahlen] Binärdarstellung $\mathsf{Bin}$: $n=\sum_{i=0}^{k-1}2^{\,k-1-i}\,x_i,\;x_i\in\{0,1\}$.
  \item[Zahlenfolgen] $(n_1,\dots,n_k)\;\mapsto\;
        \mathsf{Bin}(n_1)\,\#\,\mathsf{Bin}(n_2)\,\#\,\dots\,\#\,\mathsf{Bin}(n_k)$.
  \item[Matrizen] Matrix $(a_{ij})_{1\le i\le m}^{1\le j\le n}$ wird zeilenweise kodiert, Zeilen durch
        \texttt{\#\#}, Einträge wie oben.
\end{description}
So lassen sich komplexe Strukturen auf Wörter über $\{0,1,\#\}$ zurückführen.

%--------------------------------------------------------
\section{Abzählende Kombinatorik}\label{sec:comb}
Gegeben $n$ unterscheidbare Objekte; gesucht: Anzahl der Ziehungen von $k$ Objekten.

\renewcommand{\arraystretch}{1.2}
\begin{center}
\begin{tabular}{@{}l|c@{}}
\textbf{Situation} & \textbf{Möglichkeiten}\\\hline
geordnet, mit Zurücklegen       & $n^{k}$ \\
geordnet, ohne Zurücklegen      & $(n)_k = n(n-1)\dots(n-k+1)$ \\
ungeordnet, ohne Zurücklegen    & $\displaystyle \binom{n}{k} = \frac{n!}{k!\,(n-k)!}$
\end{tabular}
\end{center}

\noindent\textit{Beispiel:} Für $\Sigma=\{q,r,s,t\}$ seien Wörter der Länge 3 ohne Wiederholung gesucht.  
Es gilt $(4)_3 = 4\cdot3\cdot2=24$ (geordnet, ohne Zurücklegen).  
Die Formeln folgen direkt aus Produkt- und Quotientenprinzip.

%--------------------------------------------------------
\section{Automat erstellen}
\begin{enumerate}
  \item Formuliere die Sprache als Eigenschaft der bisher gelesenen Eingabe.
  \item Lege für jedes Präfix des Suchmusters einen Zustand an (z.\,B. Präfixe von \texttt{110110}: \texttt{1}, \texttt{11}, \texttt{110}, …).
  \item Definiere Übergänge so, dass jeder gelesene Buchstabe stets den längsten gültigen Präfixzustand erreicht.
  \item Markiere akzeptierende Zustände; oft genügt ein einziger.
\end{enumerate}

\paragraph{Minimalbeispiel}\leavevmode
\begin{center}
\begin{tikzpicture}[node distance=30mm,on grid,auto]
  \node[state,initial]            (q0) {$q_0$};
  \node[state,accepting,right=of q0] (q1) {$q_1$};
  \path[->]
    (q0) edge[bend left]  node {1} (q1)
    (q1) edge[bend left]  node {0} (q0)
    (q0) edge[loop above] node {0} ()
    (q1) edge[loop above] node {1} ();
\end{tikzpicture}
\end{center}
Dieser DEA akzeptiert genau die Wörter mit \emph{ungerader} Anzahl von 1ern.

%--------------------------------------------------------
\section{Automat benutzen}
Ein Wort $w$ prüfen:
\begin{enumerate}
  \item Starte in $q_0$.
  \item Lese Symbol für Symbol und folge der Übergangsfunktion $\delta$.
  \item Nach dem letzten Symbol: akzeptiere, falls aktueller Zustand in $F$, sonst verwerfe.
\end{enumerate}

%--------------------------------------------------------
% Achtung. hier muss noch der korrekte Alg aus der Vorlesung ergänzt werden. Hier wird leider der markierungs alg (nicht VL erklärt)
\section{Minimierung von DEAs}\label{sec:min-dea}
\textbf{Handlungsvorschrift (Papier):}
\begin{enumerate}
  \item \textit{Unerreichbares entfernen:} von $q_0$ aus Breadth‑First alle erreichbaren Zustände markieren, andere streichen.
  \item \textit{Tabellenmethode:}
    \begin{enumerate}
      \item Notiere jedes ungeordnete Paar $\{p,q\}$ ($p\neq q$) in einem Dreiecksgitter.
      \item Markiere sofort Paare, in denen genau einer akzeptierend ist.
      \item Wiederhole, bis stabil: markiere $(p,q)$,\\ falls ein Symbol $a$ existiert mit $\bigl(\delta(p,a),\delta(q,a)\bigr)$ bereits markiert.
    \end{enumerate}
  \item \textit{Äquivalenzklassen bilden:} unmarkierte Paare gehören zusammen $\to$ ein Zustand.
  \item \textit{Neuen DEA konstruieren:} Startzustand = Klasse von $q_0$; Übergänge entsprechend.
\end{enumerate}

\paragraph{Minimalbeispiel}\leavevmode
\begin{center}
\begin{tikzpicture}[node distance=25mm,on grid,auto]
  % Ausgangsautomat
  \node[state,initial] (A)              {$A$};
  \node[state,accepting,right=of A] (B) {$B$};
  \node[state,accepting,below=of B] (C) {$C$};
  \node[state,below=of A]           (D) {$D$};
  \path[->]
    (A) edge[bend left]  node {0} (B)
        edge[bend right] node[swap] {1} (C)
    (B) edge[loop above] node {0} ()
        edge             node {1} (C)
    (C) edge[loop right] node {1} ()
        edge             node {0} (B)
    (D) edge[loop below] node {0,1} ();
\end{tikzpicture}
\end{center}
Zustand $D$ ist unerreichbar; $B$ und $C$ verschmelzen.  
\medskip
\begin{center}
\begin{tikzpicture}[node distance=30mm,on grid,auto]
  % Minimierter Automat
  \node[state,initial]            (A)  {$\{A\}$};
  \node[state,accepting,right=of A] (BC) {$\{B,C\}$};
  \path[->]
    (A)  edge[bend left]  node {0,1} (BC)
    (BC) edge[loop above] node {0,1} ();
\end{tikzpicture}
\end{center}

%--------------------------------------------------------
\subsubsection*{Größeres Beispiel (6 Zustände, mehrstufige Minimierung)}

%------------------ Ausgangs-Automat --------------------
\begin{figure}[h]
\centering
\begin{tikzpicture}[node distance=30mm,on grid,auto]
  % Zeile 1
  \node[state,initial] (A)               {A};
  \node[state,right=of A] (B)            {B};
  \node[state,right=of B] (C)            {C};
  % Zeile 2
  \node[state,below=of A] (D)            {D};
  \node[state,accepting,below=of B] (E)  {E};
  \node[state,accepting,below=of C] (F)  {F};

  \path[->]
    % A
    (A) edge[bend right]  node{0} (B)
        edge[bend left] node[swap]{1} (C)
    % B
    (B) edge[bend right] node{0} (E)
        edge[bend right]  node{1} (C)
    % C
    (C) edge[bend left]  node[swap]{0} (E)
        edge[loop right] node{1} ()
    % D
    (D) edge[loop below] node{0,1} ()
    % E
    (E) edge[loop below] node{0} ()
        edge[bend right] node{1} (F)
    % F
    (F) edge[bend right] node[swap]{0} (E)
        edge[loop right] node[swap]{1} ();
\end{tikzpicture}
\caption{Ausgangs-DEA \(M=(Q,\Sigma,\delta,q_0,F)\) mit \(Q=\{A,\dots,F\}\)}
\end{figure}

%------------------ Schrittweise Partitionen ----------
\paragraph{Partition \(P_0\).}
Startaufteilung in akzeptierende vs.\ verwerfende Zustände:
\[
P_0=\bigl\{\underbrace{\{A,B,C,D\}}_{\text{nicht akzeptierend}}, 
          \underbrace{\{E,F\}}_{\text{akzeptierend}}\bigr\}.
\]

\paragraph{Partition \(P_1\).}
Unterscheide, wohin die Zustände auf \emph{0} wechseln:
\[
P_1=\bigl\{\{A,D\},\ \{B,C\},\ \{E,F\}\bigr\}.
\]

\paragraph{Partition \(P_2\).}
Unterscheide zusätzlich, wohin die Zustände auf \emph{1} wechseln:
\[
P_2=\bigl\{\{A\},\ \{D\},\ \{B,C\},\ \{E,F\}\bigr\}.
\]

\(P_2\) ist bereits stabil \(\Rightarrow\) \(\mathcal{M}_{\text{min}}\) hat 4 Zustände.


%------------------ Minimaler Automat -----------------
\begin{figure}[h]
\centering
\begin{tikzpicture}[node distance=22mm,on grid,auto]
  \node[state,initial]            (Amin) {$\{A\}$};
  \node[state,right=of Amin]      (BC)   {$\{B,C\}$};
  \node[state,below=of Amin]      (Dmin) {$\{D\}$};
  \node[state,accepting,right=of Dmin] (EF) {$\{E,F\}$};

  \path[->]
    % von {A}
    (Amin) edge[bend left] node{0,1} (BC)
    % von {B,C}
    (BC)   edge[bend left] node{0} (EF)
           edge[bend left] node{1} (Amin)
    % von {D}
    (Dmin) edge[loop below] node{0,1} ()
    % von {E,F}
    (EF)   edge[loop below] node{0} ()
           edge[loop right] node{1} ();
\end{tikzpicture}
\caption{Ergebnis der Minimierung: 4‐Zustands-DEA \(\mathcal{M}_{\text{min}}\)}
\end{figure}

\medskip
\noindent
Damit sieht man Schritt für Schritt, wie das Äquivalenz­klassen-Verfahren 
(Erdös–Myrvold / Moore / Hopcroft) arbeitet und letztlich die Übergänge des 
minimalen Automaten erhält.


%--------------------------------------------------------
\section{Verknüpfung von Automaten}
Union zweier Automaten $M_1,M_2$ ($L=L(M_1)\cup L(M_2)$):
\begin{enumerate}
  \item Füge neuen Startzustand $S$ ein.
  \item Verbinde $S$ per $\varepsilon$-Kanten mit den Startzuständen von $M_1$ und $M_2$ ($\varepsilon$‑NEA).
  \item Übernehme alle Zustände und Transitionen beider Automaten.
  \item Akzeptierend sind sämtliche akzeptierenden Zustände aus $M_1$ und $M_2$.
\end{enumerate}

\paragraph{Minimalbeispiel (Vereinigung)}\leavevmode
\begin{center}
\begin{tikzpicture}[node distance=22mm,on grid,auto]
  % DEA M1 – endet auf a
  \node[state,initial]            (p0) {$p_0$};
  \node[state,right=of p0]        (p1) {$p_1$};
  \node[state,accepting,right=of p1] (p2) {$p_2$};
  \path[->]
    (p0) edge node {a} (p1)
         edge[loop above] node {b} ()
    (p1) edge node {a} (p2)
         edge[loop above] node {b} ()
    (p2) edge[loop above] node {a,b} ();
  % DEA M2 – endet auf b
  \node[state,initial,below=30mm of p0] (q0) {$q_0$};
  \node[state,right=of q0]              (q1) {$q_1$};
  \node[state,accepting,right=of q1]    (q2) {$q_2$};
  \path[->]
    (q0) edge node {b} (q1)
         edge[loop below] node {a} ()
    (q1) edge node {b} (q2)
         edge[loop below] node {a} ()
    (q2) edge[loop below] node {a,b} ();
  % gemeinsamer Start S
  \node[state,initial,above right=of q1] (S) {$S$};
  \path[->]
    (S) edge[dashed] node {$\varepsilon$} (p0)
    (S) edge[dashed] node {$\varepsilon$} (q0);
\end{tikzpicture}
\end{center}

\noindent\textbf{Schnitt / allgemeine Boolesche Operationen}  
Für den \emph{Schnitt} $L=L(M_1)\cap L(M_2)$ (ebenso XOR, Differenz usw.) baut man einen
\emph{Produktautomaten}:
\begin{itemize}
  \item \textit{Zustände:} kartesische Paare $(p,q)$ mit $p\in Q_1,\;q\in Q_2$.
  \item \textit{Startzustand:} $(q_{0,1},q_{0,2})$.
  \item \textit{Übergang:} $\delta((p,q),a)=(\delta_1(p,a),\delta_2(q,a))$.
  \item \textit{Akzeptanz:} beide Komponenten akzeptierend (Schnitt). Für Vereinigung mindestens eine, für Differenz nur die erste usw.
\end{itemize}

\paragraph{Beispiel (Schnitt)}\leavevmode
\begin{center}
% Zwei kleine DEAs und ihr Produktautomat (Schnitt)
\begin{tikzpicture}[node distance=22mm,on grid,auto]
  % M1: gerade Anzahl 0
  \node[state,initial]            (a0) {$p_0$};
  \node[state,accepting,right=of a0] (a1) {$p_1$};
  \path[->]
    (a0) edge[bend left]  node {0} (a1)
    (a1) edge[bend left]  node {0} (a0)
    (a0) edge[loop above] node {1} ()
    (a1) edge[loop above] node {1} ();
  % M2: endet auf 0
  \node[state,initial,below=30mm of a0] (b0) {$q_0$};
  \node[state,accepting,right=of b0]    (b1) {$q_1$};
  \path[->]
    (b0) edge node {0} (b1)
         edge[loop below] node {1} ()
    (b1) edge[loop below] node {0} ()
         edge[bend left] node {1} (b0);
\end{tikzpicture}

\vspace{1em}

\begin{tikzpicture}[node distance=28mm,on grid,auto]
  % Produktautomat für den Schnitt
  \node[state,initial]                 (s00) {$\langle p_0,q_0\rangle$};
  \node[state,right=of s00]            (s10) {$\langle p_1,q_0\rangle$};
  \node[state,below=of s00]            (s01) {$\langle p_0,q_1\rangle$};
  \node[state,accepting,right=of s01]  (s11) {$\langle p_1,q_1\rangle$};

  \path[->]
    % von s00
    (s00) edge[bend right] node {0} (s10)
          edge[loop above] node[swap]{1} ()
    % von s10
    (s10) edge[bend right] node {0} (s00)
          edge[loop above] node{1} ()
    % von s01
    (s01) edge[bend right] node[swap] {0} (s11)
          edge node[swap] {1} (s00)
    % von s11
    (s11) edge[bend right] node {0} (s01)
          edge node {1} (s10);
\end{tikzpicture}
\end{center}
Der Produktautomat akzeptiert genau die Wörter, die sowohl eine \emph{gerade Anzahl von~0} besitzen als auch auf~0 enden.  
Damit illustriert das Beispiel anschaulich, wie der Schnitt zweier Sprachen mittels kartesischem Produkt realisiert wird.

%--------------------------------------------------------
\section{\texorpdfstring{$L^{R}$ mit $\varepsilon$-NEA erkennen}{LR mit ε-NEA erkennen}}
Gegeben ein DEA $M$ für $L$, konstruiere einen $\varepsilon$-NEA $M'$ für die Umkehrsprache $L^R$:
\begin{enumerate}
  \item \textit{Kanten umdrehen:} jede Transition $p\xrightarrow{a}q$ wird $q\xrightarrow{a}p$.
  \item \textit{Zustände tauschen:} alter Startzustand wird allein akzeptierend.
  \item \textit{Neuer Startzustand $S$:} $\varepsilon$-Kanten von $S$ zu allen ehemaligen akzeptierenden Zuständen.
\end{enumerate}
$M'$ akzeptiert genau dann, wenn $M$ das rückwärts gelesene Wort akzeptiert.

\paragraph{Minimalbeispiel}\leavevmode
\begin{center}
\begin{tikzpicture}[node distance=32mm,on grid,auto]
  % ursprünglicher DEA M für L = {ab}
  \node[state,initial]           (r0) {$r_0$};
  \node[state,right=of r0]       (r1) {$r_1$};
  \node[state,accepting,right=of r1] (r2) {$r_2$};
  \path[->]
    (r0) edge node {a} (r1)
    (r1) edge node {b} (r2);
\end{tikzpicture}
\qquad
\begin{tikzpicture}[node distance=32mm,on grid,auto]
  % reversed ε-NEA M'
  \node[state,initial] (s) {$S$};
  \node[state,accepting,right=of s] (p0) {$r_0$};
  \node[state,right=of p0] (p1) {$r_1$};
  \node[state,right=of p1] (p2) {$r_2$};
  \path[->]
    (s) edge[dashed, bend right] node {$\varepsilon$} (p2)
    (p2) edge node {b} (p1)
    (p1) edge node {a} (p0);
\end{tikzpicture}
\end{center}

%--------------------------------------------------------
\section{\texorpdfstring{NEA $\Rightarrow$ DEA: Potenzmengenkonstruktion}{NEA → DEA: Potenzmengenkonstruktion}}
\textbf{Schritte (Papier):}
\begin{enumerate}
  \item \textit{Startzustand:} $\{q_0\}$.
  \item \textit{Zustände:} alle erreichbaren Teilmengen $S\subseteq Q_N$.
  \item \textit{Übergang:} $\delta_D(S,a)=\bigcup_{q\in S} \delta_N(q,a)$.
  \item \textit{Akzeptierend:} $S$ ist akzeptierend, falls $S\cap F_N\neq\varnothing$.
\end{enumerate}

\paragraph{Minimalbeispiel}\leavevmode
\begin{center}
% Ursprünglicher NEA
\begin{tikzpicture}[node distance=28mm,on grid,auto]
  \node[state,initial]            (nA) {$A$};
  \node[state,right=of nA]        (nB) {$B$};
  \node[state,accepting,below=of nA] (nC) {$C$};
  \path[->]
    (nA) edge[loop above] node {0} ()
         edge[bend left]  node {1} (nB)
    (nB) edge[bend left]  node[swap] {0,1} (nC)
    (nC) edge[loop below] node {0,1} ();
\end{tikzpicture}
\end{center}
Dieser NEA akzeptiert Wörter, in denen mindestens ein Symbol~1 auf eine beliebige~0 folgt.

\medskip
\noindent
Nach Anwendung der Potenzmengenkonstruktion erhält man folgende erreichbare Zustände (nur vier von $2^{3}=8$ sind tatsächlich nötig):
\[
\begin{array}{c|c|c}
S & 0 & 1\\\hline
\{A\} & \{A\} & \{A,B\} \\
\{A,B\} & \{A,C\} & \{A,B,C\} \\
\{A,C\} & \{A\} & \{A,B\} \\
\{A,B,C\} & \{A,C\} & \{A,B,C\}
\end{array}
\]
\begin{center}
% DEA nach PMK
\begin{tikzpicture}[node distance=32mm,on grid,auto]
  \node[state,initial]                 (d0) {$\{A\}$};
  \node[state,right=of d0]             (d1) {$\{A,B\}$};
  \node[state,below=of d0]             (d2) {$\{A,C\}$};
  \node[state,accepting,right=of d2]   (d3) {$\{A,B,C\}$};
  \path[->]
    (d0) edge[bend left]      node {1} (d1)
         edge[loop above]     node {0} ()
    (d1) edge[bend left]      node {0} (d2)
         edge[bend left]      node {1} (d3)
    (d2) edge[bend left]      node {1} (d1)
         edge[bend left]      node {0} (d0)
    (d3) edge[loop below]     node {1} ()
         edge[bend left]      node {0} (d2);
\end{tikzpicture}
\end{center}
Die akzeptierenden DEA‑Zustände sind genau die Teilmengen, die $C$ enthalten (hier: $\{A,C\}$ und $\{A,B,C\}$).  
Das Beispiel demonstriert ausführlich das exponentielle Wachstum der Zustandsanzahl und verdeutlicht die auf den Folien gezeigte Tabelle von Prof.~Striegnitz.  
\medskip
\textbf{Faustregel:} Beim Zeichnen genügt es stets, nur die \emph{erreichbaren} Teilmengen zu betrachten – das spart Zeit und Tinte.

%========================================================

% chapters/laufzeit.tex – Laufzeit von Algorithmen
% (benötigt keine zusätzlichen Pakete außer denen aus main.tex)

\chapter{Laufzeit von Algorithmen}

%--------------------------------------------------------
\section{\texorpdfstring{Definition von $O,\Omega,\Theta$}{Definition von O, Ω, Θ}}
\subsection*{Mathematische Formulierung}
Seien $f,g\colon \mathbb N \to \mathbb R_{>0}$ Funktionen. Dann definiert die
\emph{Landau-Notation} die Mengen
\begin{align*}
  O(f)      &:= \{\,g \mid \exists\,c,n_0>0: \forall n\ge n_0:\; g(n)\le c\,f(n)\,\},\\
  \Omega(f) &:= \{\,g \mid \exists\,c,n_0>0: \forall n\ge n_0:\; g(n)\ge c\,f(n)\,\},\\
  \Theta(f) &:= O(f)\cap\Omega(f).
\end{align*}
Kurz: $g\in O(f)$ liefert eine \emph{obere Schranke}, $g\in\Omega(f)$ eine
\emph{untere Schranke}, und $g\in\Theta(f)$ heißt, daß $f$ und $g$ asymptotisch
gleiche Wachstumsordnung besitzen.

\subsection*{Umgangssprachliche Faustformel}
\begin{description}
  \item[$O$ („O-groß“)] gibt an, wie schnell \emph{spätestens} etwas wächst – eine Garantie nach oben.
  \item[$\Omega$ („Omega-groß“)] gibt an, wie schnell \emph{mindestens} etwas wächst – eine Garantie nach unten.
  \item[$\Theta$ („Theta-groß“)] bedeutet „genau in dieser Größenordnung“ – die Funktion wächst weder wesentlich schneller noch wesentlich langsamer.
\end{description}

%--------------------------------------------------------
\section{Funktionen asymptotisch vergleichen}
Um zwei Funktionen $f$ und $g$ zu vergleichen, prüft man oft den Grenzwert
\[\lim_{n\to\infty} \frac{f(n)}{g(n)}\;=\;\begin{cases}
  0 &\Rightarrow f\in O(g)\\[4pt]
  \infty &\Rightarrow f\in \Omega(g)\\[4pt]
  c\in(0,\infty) &\Rightarrow f\in \Theta(g).
\end{cases}\]
Typische Wachstumshierarchie:
\[1\;\ll\; \log\!n\;\ll\; n^k\;\ll\; n^k\log n\;\ll\; a^n\;\ll\; n!\;\ll\; n^{n}.\]
Ein schneller Test ist daher, Polynom- gegen Exponential- oder Fakultäts-Term
zu vergleichen; stetig vorliegende $\log$-Faktoren spielen dabei nur eine
untergeordnete Rolle.

%--------------------------------------------------------
\section{Master-Theorem}
Betrachte Rekursionen der Form
\[T(n)=a\,T\!\bigl(\tfrac{n}{b}\bigr)+d(n)\qquad(a\ge1,\;b>1).\]
Sei $d(n)\in O(n^{\gamma})$ für ein $\gamma\ge0$. Dann gilt
\[
T(n)=\begin{cases}
  \Theta\bigl(n^{\log_b a}\bigr), & a> b^{\gamma},\\[4pt]
  \Theta\bigl(n^{\gamma}\log n\bigr), & a= b^{\gamma},\\[4pt]
  \Theta\bigl(n^{\gamma}\bigr), & a< b^{\gamma}.
\end{cases}
\]
\subsection*{Beispiele}
\begin{enumerate}
  \item[Fall 1:] $T(n)=2T(n/2)+1$\\
        $a=2,\;b=2,\;d(n)=1\in O(n^{0})$ mit $a>b^{0}=1$ $\Rightarrow$
        $T(n)=\Theta(n^{\log_22})=\Theta(n)$.\\
        (Lineare Zeit, etwa beim Best-case von Quicksort.)
  \item[Fall 2:] $T(n)=2T(n/2)+n$\\
        $a=2,\;b=2,\;d(n)=n$ mit $\gamma=1$ und $a=b^{\gamma}$ $\Rightarrow$
        $T(n)=\Theta(n\log n)$.\\
        (Klassisch: Mergesort.)
  \item[Fall 3:] $T(n)=9T(n/3)+n$\\
        $a=9,\;b=3,\;d(n)=n$ mit $\gamma=1$ und $a>b^{\gamma}=3$ $\Rightarrow$
        $T(n)=\Theta\bigl(n^{\log_39}\bigr)=\Theta(n^{2})$.\\
        (Beispiel aus Strassen-Matrixmultiplikation.)
\end{enumerate}

%--------------------------------------------------------
\section{Erzeugende Funktionen}
Gegeben eine Folge $(a_n)_{n\ge0}$ definiert die
\emph{erzeugende Funktion}
\[F(x)=\sum_{n=0}^{\infty} a_n x^{n}.\]
Vorteil: Verschiebungen werden zu einfachen Operationen, z.\,B.\
$x\,F(x)=\sum_{n\ge0} a_{n-1}x^{n}$ (Index-Shift).

\subsection*{Idee}
Eine Rekursion für $a_n$ wird in eine Gleichung für $F(x)$ übersetzt, diese nach
$F(x)$ aufgelöst und anschließend über Potenzreihenentwicklung $a_n$
ausgelesen.

\subsection*{Beispiel: Geometrische Folge}
Sei $a_n=2^n$ mit $a_0=1$. Dann
\[F(x)=\sum_{n\ge0} (2x)^n=\frac{1}{1-2x},\qquad |x|<\tfrac{1}{2}.\]
Die geschlossene Form verrät Radius der Konvergenz $\tfrac12$ und bestätigt
$a_n=[x^{n}]F(x)=2^{n}$.

\subsection*{Beispiel: Fibonacci}
Für $F(x)=\sum_{n\ge0} \text{Fib}(n) x^n$ gilt wegen $\text{Fib}(n)=\text{Fib}(n-1)+\text{Fib}(n-2)$
\[F(x)=x + xF(x) + x^{2}F(x) \;\Longrightarrow\; F(x)=\frac{x}{1 - x - x^{2}}.\]
Eine Partialbruchzerlegung liefert schließlich die Binet-Formel
\[\text{Fib}(n)=\frac{\varphi^{n}-\psi^{n}}{\sqrt{5}},\qquad
  \varphi=\tfrac{1+\sqrt{5}}{2},\;
  \psi=1-\varphi.\]

% chapters/algdatgrundlagen.tex – Algorithmen & Datenstrukturen – Grundlagen
% Benötigt im Hauptdokument keine zusätzlichen Pakete neben main.tex

% -----------------------------------------------------------------------------
\chapter{Algorithmen \& Datenstrukturen – Grundlagen}

% ##############################################################################
% 1. VERKETTETE LISTEN
% ##############################################################################
\section{Umgang mit verketteten Listen}
\subsection*{Grundidee}
Eine \emph{einfach verkettete Liste} besteht aus Knoten der Form
\[\texttt{\{ key, next \}}\].
Im Gegensatz zu Arrays wächst die Struktur dynamisch, weil Einfügen/Löschen
nur Zeiger (bzw. Referenzen) ändert.

\subsection*{Typische Operationen und Laufzeiten}
\begin{center}
  \renewcommand{\arraystretch}{1.15}
  \begin{tabular}{@{}lcc@{}}
    \textbf{Operation} & \textbf{am Kopf} & \textbf{allg. Position}\\\hline
    \textsc{insert}   & $\Theta(1)$   & $\Theta(1)$\,*\\
    \textsc{delete}   & $\Theta(1)$   & $\Theta(1)$\,*\\
    \textsc{search}   & \multicolumn{2}{c}{$\Theta(n)$}\\
  \end{tabular}
\end{center}
\emph{\small *\,Vorausgesetzt, ein Verweis auf den Vorgängerknoten ist gegeben.}

\subsection*{Beispielimplementierung (Java)}
\begin{verbatim}
static class Node {
    int key;
    Node next;
    Node(int k, Node n) { key = k; next = n; }
}

static Node pushFront(Node head, int x) {
    return new Node(x, head);
}

static Node popFront(Node head) {
    return (head == null) ? null : head.next;
}
\end{verbatim}
Doppelt verkettete Listen ergänzen ein Feld \texttt{prev} und erlauben damit
$\Theta(1)$ Löschungen ohne Vorgängerzugriff.

\vspace{1em}\noindent
\textbf{Praxisfaustregel:} \emph{Listen für häufige Ein-/Ausfügeoperationen,
Arrays für zufälligen Direktzugriff.}

% ##############################################################################
% 2. SIMULATION EINER DS MIT EINER ANDEREN
% ##############################################################################
\section{Simulation einer Datenstruktur mit einer anderen}
\subsection*{Queue mit zwei Stacks}
\begin{algorithmic}[H]
\Require Zwei Stacks $S_{\text{in}},\,S_{\text{out}}$
\Function{enqueue}{$x$} \Comment{immer $\Theta(1)$}
    push $x$ auf $S_{\text{in}}$
\EndFunction
\Function{dequeue}{} \Comment{amortisiert $\Theta(1)$}
    \If{$S_{\text{out}}$ leer}
        \While{$S_{\text{in}}$ nicht leer} \Comment{jedes Element max. 1\,$\times$ verschoben}
            verschiebe oberstes Element nach $S_{\text{out}}$
        \EndWhile
    \EndIf
    return pop von $S_{\text{out}}$
\EndFunction
\end{algorithmic}

\subsection*{Stack mit zwei Queues}
\textbf{Lazy‑Variante} (\textsc{push} billig, \textsc{pop} teuer):
\begin{algorithmic}[H]
\Require Zwei Queues $Q_1,\,Q_2$
\Function{push}{$x$}
    enqueue $x$ in $Q_1$  \Comment{$\Theta(1)$}
\EndFunction
\Function{pop}{}
    \While{$|Q_1| > 1$} verschiebe Front von $Q_1$ nach $Q_2$ \EndWhile
    swap $Q_1, Q_2$; return letztes Element \Comment{$\Theta(n)$}
\EndFunction
\end{algorithmic}
\textbf{Eager‑Variante} dreht das Kostenprofil um.

\subsection*{Cursor‑Listen (Array \texorpdfstring{$\leftrightarrow$}{↔} Liste)}
Ein Array simuliert Zeiger durch Indizes – nützlich, wenn echte Zeiger verboten
oder Speicher abstrahiert wird (z.\,B. Prüfungen).

% ##############################################################################
% 3. BAUMPARAMETER
% ##############################################################################
\section{Baumparameter berechnen}
\subsection*{Definitionen}
\begin{description}
  \item[Größe] $|B|$ … Anzahl Knoten.
  \item[Blätter] $\mathrm{leaf}(B)$ … Knoten Grad $0$.
  \item[Höhe] $h(B)$ … Länge des längsten Pfades Wurzel $\rightarrow$ Blatt.
\end{description}

\subsection*{Rekursive Berechnung (Pseudocode)}
\begin{verbatim}
size(v):
  if v == null return 0
  return 1 + size(v.left) + size(v.right)

treeHeight(v):                leafCount(v):
  if v == null return -1        if v == null return 0
  return 1 + max(               if v.left==null && v.right==null return 1
      treeHeight(v.left),       return leafCount(v.left)
      treeHeight(v.right))                + leafCount(v.right)
\end{verbatim}
Alle Algorithmen traversieren jeden Knoten einmal: $\Theta(n)$ Zeit und
$\Theta(h)$ Zusatzspeicher durch den Rekursionsstapel.

\subsection*{Grenzen für die Höhe}
Für max. Kind‐Grad $d$ und $n$ Knoten gilt
\[\boxed{\;\lceil\log_d n\rceil \;\le\; h(B) \;\le\; n-1\;}\]

% ##############################################################################
% 4. BAUMDURCHLÄUFE
% ##############################################################################
\section{Baumdurchläufe (inkl. Level‑order)}
\begin{enumerate}
  \item \textbf{Preorder}: Wurzel $\rightarrow$ links $\rightarrow$ rechts
  \item \textbf{Inorder}: links $\rightarrow$ Wurzel $\rightarrow$ rechts (BST $\rightarrow$ sortiert)
  \item \textbf{Postorder}: links $\rightarrow$ rechts $\rightarrow$ Wurzel
  \item \textbf{Level‑order} (BFS): Ebenenweise links $\rightarrow$ rechts
\end{enumerate}

\subsection*{Iterative Level‑order in Java}
\begin{verbatim}
static void levelOrder(Node root) {
    Queue<Node> q = new ArrayDeque<>();
    q.add(root);
    while (!q.isEmpty()) {
        Node v = q.remove();
        visit(v);
        if (v.left  != null) q.add(v.left);
        if (v.right != null) q.add(v.right);
    }
}
\end{verbatim}
Laufzeit $\Theta(n)$, Speicher $\Theta(w)$ mit $w$ = maximale Breite.

% ##############################################################################
% 5. ALGORITHMENANSÄTZE ERKENNEN
% ##############################################################################
\section{Ansatz eines Algorithmus erkennen}
\begin{description}
  \item[Divide \& Conquer] rekursive Zerlegung (z.\,B. Mergesort).
  \item[Backtracking] Tiefensuche mit Zurückrudern (z.\,B. 8‑Damen‑Problem).
  \item[Dynamische Programmierung] überlappende Teilprobleme, Tabellierung.
  \item[Greedy] Wahl lokal optimaler Schritte ohne spätere Korrektur
        (z.\,B. Kruskal‑MST).
\end{description}
Charakteristisch sind typische Code‑Gerüste: Rekursion + Merge, Tabellenschleife,
fortlaufender "beste Wahl"‑Schritt…

% ##############################################################################
% 6. ALGORITHMUS ENTWICKELN
% ##############################################################################
\section{Algorithmus aus Vorgaben entwickeln}
\begin{enumerate}
  \item \textbf{Problem analysieren}: Eingabe/Output exakt definieren.
  \item \textbf{Strategie wählen}: D\&C, DP, Greedy, Backtracking…
  \item \textbf{Datenstrukturen}: Array, Liste, Heap, Graph…
  \item \textbf{Pseudocode} \& Invarianten festhalten.
  \item \textbf{Komplexität}: Landau, evtl. Master‑Theorem für Rekursionen.
\end{enumerate}

% ##############################################################################
% 7. UNPASSENDEN ANSATZ BEGRÜNDEN
% ##############################################################################
\section{Unpassenden Ansatz begründen / Gegenbeispiel finden}
\subsection*{Vorgehen}
\begin{enumerate}
  \item Kernannahme des Ansatzes identifizieren.
  \item Eingabe konstruieren, die Annahme widerlegt.
  \item Falsche Ausgabe oder höhere Kosten nachweisen.
\end{enumerate}
\subsection*{Beispiel: Greedy beim Rucksackproblem}
Greedy wählt stets maximales \emph{Wert/Gewicht}.  Gegenbeispiel:
\[
  (w,v)=\{(5,9),(6,10),(1,1)\},\;C=7
\]
Greedy nimmt $(5,9)+(1,1)$ (Wert 10), optimal ist $(6,10)$.


% chapters/sortierverfahren.tex – Sortierverfahren (mit Beispielen)
% Benötigt im Hauptdokument nur die Pakete aus main.tex (algorithm + algpseudocode
% wurden dort bereits ergänzt).

\chapter{Sortierverfahren}

% =========================================================================
% 1. INSERTION SORT
% =========================================================================
\section{Insertion‑Sort anwenden}
\subsection*{Idee}
Ein elementares, stabiles In‑Situ‑Verfahren: durchläuft das Feld von links
nach rechts und fügt jedes neue Element an der korrekten Stelle in den bereits
\emph{sortierten} linken Teil ein – wie das Sortieren eines Kartenstapels.

\subsection*{Algorithmus (Java)}
\begin{algorithmic}
\Function{insertionSort}{int[] a}
  \For{$i\gets1$ \textbf{to} $a.\text{length}-1$}
    \State $key \gets a[i]$; $j \gets i$;
    \While{$j>0$ \textbf{and} $a[j-1]>key$}
         \State $a[j] \gets a[j-1]$; $j--$;
    \EndWhile
    \State $a[j] \gets key$;
  \EndFor
\EndFunction
\end{algorithmic}

\subsubsection*{Beispiel}
Eingabe \([4,\,3,\,1,\,5,\,2]\)
\[
\begin{array}{l|lllll}
\text{Schritt}&1&3&1&5&2\\\hline
\text{Nach $i=1$}&3&4&1&5&2\\
\text{Nach $i=2$}&1&3&4&5&2\\
\text{Nach $i=3$}&1&3&4&5&2\\
\text{Nach $i=4$}&1&2&3&4&5
\end{array}
\]

\subsection*{Laufzeit und Eigenschaften}
\begin{itemize}
  \item Best Case $O(n)$ bei bereits sortierten Daten.\par
  \item Worst / Average Case $O(n^{2})$ Vergleiche und Verschiebungen.\par
  \item Speicher $O(1)$, stabil.\par
  \item Sehr effizient für kleine oder fast sortierte Eingaben.
\end{itemize}

% =========================================================================
% 2. QUICK SORT
% =========================================================================
\section{Quick‑Sort anwenden}
\subsection*{Idee}
Divide \& Conquer: Wähle ein \emph{Pivot}, partitioniere das Feld in Werte
<Pivot und >Pivot, sortiere Teilfelder rekursiv; verknüpfe in‑place.

% Partitionierung prüfen
\subsection*{Algorithmus (Java – Hoare‑Partition, Rekursion)}
\begin{algorithmic}
\Function{quickSort}{int[] a, int l, int r}
  \If{$l<r$}
    \State $p \gets \Call{partition}{a,l,r}$; \Comment{Rückgabe = Pivot‑Index}
    \State \Call{quickSort}{a,l,p-1};
    \State \Call{quickSort}{a,p+1,r};
  \EndIf
\EndFunction
\end{algorithmic}

\subsubsection*{Beispiel (erstes Element als Pivot)}
Eingabe \([4,\,3,\,1,\,5,\,2]\)
\begin{enumerate}[label=\alph*)]
  \item Pivot 4 $\rightarrow$ Partition: \([3,1,2]\,4\,[5]\)
  \item Rek. links: Pivot $\rightarrow$ \([1,2]\,3\,[]\).  Rek. rechts sortiert (1 Elem).
  \item Rek. auf \([1,2]\): Pivot $\rightarrow$ \([]\,1\,[2]\).
  \item Zusammengeführt: \([1,2,3,4,5]\)
\end{enumerate}

\subsection*{Laufzeit und Eigenschaften}
\begin{itemize}
  \item Average Case $O(n\log n)$, Worst Case $O(n^{2})$ bei ungünstigen Pivots.\par
  \item Speicher $O(\log n)$ Rekursionstiefe, instabil, in‑place.\par
  \item Meist schnellstes praktisches Vergleichsverfahren.
\end{itemize}

% =========================================================================
% 3. HEAP VERWENDEN
% =========================================================================
\section{Heap verwenden}
\subsection*{Definition}
Ein \emph{Heap} ist ein vollständiger Binärbaum, der die Heap‑Eigenschaft erfüllt.

\subsection*{Kernoperationen (Pseudocode)}
\begin{algorithmic}
\Function{insert}{int x}
  \State $heap[++size] \gets x$; \Call{shiftUp}{size}
\EndFunction
\Function{extractMax}{}
  \State $max \gets heap[1]$; $heap[1] \gets heap[size--]$;
  \State \Call{shiftDown}{1}; \Return $max$;
\EndFunction
\end{algorithmic}

\subsubsection*{Beispiel – Max‑Heap Aufbau}
Eingabe \([4,\,3,\,1,\,5,\,2]\);
\begin{center}
\begin{tabular}{c|c}
Einfügen & Heap‑Array (1‑basiert)\\\hline
4 & 4\\
3 & 4\,3\\
1 & 4\,3\,1\\
5 & 5\,4\,1\,3\\
2 & 5\,4\,2\,3\,1
\end{tabular}
\end{center}

% =========================================================================
% 4. HEAP SORT
% =========================================================================
\section{Heap‑Sort anwenden}
\subsection*{Idee}
\textbf{Build‑Heap} und dann wiederholt Maximum nach hinten tauschen.

\subsubsection*{Beispiel}
Start‑Heap aus vorherigem Beispiel. Entferne iterativ Maximum:
\[
5\,\underline{4\,2\,3\,1}\;\rightarrow\;1\,\underline{4\,2\,3}\;\rightarrow\;1\,2\,\underline{3}\;\rightarrow\;1\,2\,3\,4\,5
\]

\subsection*{Eigenschaften}
Laufzeit garantiert $O(n\log n)$, Speicher $O(1)$, instabil.

% =========================================================================
% 5. MERGE SORT
% =========================================================================
\section{Merge‑Sort anwenden}
\subsection*{Idee}
Rekursiv halbieren, dann zusammenführen.

\subsubsection*{Beispiel (Top‑Down)}
\begin{align*}
[4,3,1,5,2] &\Rightarrow [4,3] \,[1,5,2]\\
&\Rightarrow [4] [3] \;[1] [5,2]\\
&\Rightarrow [2,5] (Merge)\\
&\Rightarrow [3,4] (Merge)\\
&\Rightarrow [1,2,5] (Merge)\\
&\Rightarrow [1,2,3,4,5] (Final‑Merge)
\end{align*}

\subsection*{Eigenschaften}
Immer $O(n\log n)$, Speicher $O(n)$, stabil.

% =========================================================================
% 6. COUNTING SORT
% =========================================================================
\section{Counting‑Sort anwenden}
\subsection*{Beispiel}
Eingabe \([4,3,1,5,2] ,\;k=6\)
\[
C=\begin{array}{c|cccccc}
val&0&1&2&3&4&5\\\hline
count&0&1&1&1&1&1\end{array}
\;
\Rightarrow\;\text{Prefix}=[0,1,2,3,4,5]\;
\Rightarrow\;\text{Output}=[1,2,3,4,5]
\]

\subsection*{Komplexität}
$O(n+k)$ Zeit/Speicher, stabil.

% =========================================================================
% 7. BUCKET SORT
% =========================================================================
\section{Bucket‑Sort anwenden}
\subsection*{Beispiel}
Normiere Array \([0.42,0.32,0.23,0.52,0.12]\) mit $m=5$ Buckets.
\begin{center}
\begin{tabular}{c|l}
Bucket & Elemente nach Streuung / Sortierung\\\hline
0 & 0.12\\
1 & 0.23\,0.32\\
2 & 0.42\\
3 & 0.52\\
4 & –
\end{tabular}
\end{center}
Konkatenation ergibt \([0.12,0.23,0.32,0.42,0.52]\).

\subsection*{Komplexität}
Average $O(n)$, Worst $O(n^{2})$.

% =========================================================================
% 8. RADIX SORT
% =========================================================================
\section{Radix‑Sort anwenden}
\subsection*{Beispiel (Dezimal, LSD, $d=2$)}
Eingabe \([23,17,45,11,29]\)
\begin{enumerate}[label=Stelle~\arabic*]
  \item Einer‑Sortierung → \([11,23,45,17,29]\)
  \item Zehner‑Sortierung → \([11,17,23,29,45]\)
\end{enumerate}

\subsection*{Komplexität}
$O(d(n+b))$ Zeit.

\chapter{Suchen}

\section{Binäre Suche anwenden}

\section{Gegebene Hashfunktion benutzen}

\section{Sondierungsmethode anwenden}

\section{Binäre Suchbäume ein-/ausfügen}

\section{Leveldifferenz bei AVL-Bäumen bestimmen}

\section{AVL-Baum ein-/ausfügen}

\section{B-Baum ein-/ausfügen}

\section{Größter Rand $\pi[i]$ bestimmen}

\section{KMP-Verfahren anwenden}

\section{Boyer-Moore (einfache Bad-Character) anwenden}


\chapter{Algorithmen auf Graphen}

\section{BFS / DFS anwenden}

\section{DAG topologisch sortieren}

\section{Dijkstra-Algorithmus anwenden}

\section{Bellman–Ford-Algorithmus anwenden}

\section{Floyd–Warshall-Algorithmus anwenden}

\section{Ford–Fulkerson-Algorithmus anwenden}

\section{Fluss → Matching reduzieren}

\section{Kruskal-Algorithmus anwenden}

\section{Prim-Algorithmus anwenden}

\section{2-Approximation für TSP anwenden}

\section{1,5-Approximation von Christofides anwenden}



\end{document}
