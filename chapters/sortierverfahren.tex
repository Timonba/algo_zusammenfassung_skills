% chapters/sortierverfahren.tex – Sortierverfahren (mit Beispielen)
% Benötigt im Hauptdokument nur die Pakete aus main.tex (algorithm + algpseudocode
% wurden dort bereits ergänzt).

\chapter{Sortierverfahren}

% =========================================================================
% 1. INSERTION SORT
% =========================================================================
\section{Insertion‑Sort anwenden}
\subsection*{Idee}
Ein elementares, stabiles In‑Situ‑Verfahren: durchläuft das Feld von links
nach rechts und fügt jedes neue Element an der korrekten Stelle in den bereits
\emph{sortierten} linken Teil ein – wie das Sortieren eines Kartenstapels.

\subsection*{Algorithmus (Java)}
\begin{algorithmic}
\Function{insertionSort}{int[] a}
  \For{$i\gets1$ \textbf{to} $a.\text{length}-1$}
    \State $key \gets a[i]$; $j \gets i$;
    \While{$j>0$ \textbf{and} $a[j-1]>key$}
         \State $a[j] \gets a[j-1]$; $j--$;
    \EndWhile
    \State $a[j] \gets key$;
  \EndFor
\EndFunction
\end{algorithmic}

\subsubsection*{Beispiel}
Eingabe \([4,\,3,\,1,\,5,\,2]\)
\[
\begin{array}{l|lllll}
\text{Schritt}&1&3&1&5&2\\\hline
\text{Nach $i=1$}&3&4&1&5&2\\
\text{Nach $i=2$}&1&3&4&5&2\\
\text{Nach $i=3$}&1&3&4&5&2\\
\text{Nach $i=4$}&1&2&3&4&5
\end{array}
\]

\subsection*{Laufzeit und Eigenschaften}
\begin{itemize}
  \item Best Case $O(n)$ bei bereits sortierten Daten.\par
  \item Worst / Average Case $O(n^{2})$ Vergleiche und Verschiebungen.\par
  \item Speicher $O(1)$, stabil.\par
  \item Sehr effizient für kleine oder fast sortierte Eingaben.
\end{itemize}

% =========================================================================
% 2. QUICK SORT
% =========================================================================
\section{Quick‑Sort anwenden}
\subsection*{Idee}
Divide \& Conquer: Wähle ein \emph{Pivot}, partitioniere das Feld in Werte
<Pivot und >Pivot, sortiere Teilfelder rekursiv; verknüpfe in‑place.

% Partitionierung prüfen
\subsection*{Algorithmus (Java – Hoare‑Partition, Rekursion)}
\begin{algorithmic}
\Function{quickSort}{int[] a, int l, int r}
  \If{$l<r$}
    \State $p \gets \Call{partition}{a,l,r}$; \Comment{Rückgabe = Pivot‑Index}
    \State \Call{quickSort}{a,l,p-1};
    \State \Call{quickSort}{a,p+1,r};
  \EndIf
\EndFunction
\end{algorithmic}

\subsubsection*{Beispiel (erstes Element als Pivot)}
Eingabe \([4,\,3,\,1,\,5,\,2]\)
\begin{enumerate}[label=\alph*)]
  \item Pivot 4 $\rightarrow$ Partition: \([3,1,2]\,4\,[5]\)
  \item Rek. links: Pivot $\rightarrow$ \([1,2]\,3\,[]\).  Rek. rechts sortiert (1 Elem).
  \item Rek. auf \([1,2]\): Pivot $\rightarrow$ \([]\,1\,[2]\).
  \item Zusammengeführt: \([1,2,3,4,5]\)
\end{enumerate}

\subsection*{Laufzeit und Eigenschaften}
\begin{itemize}
  \item Average Case $O(n\log n)$, Worst Case $O(n^{2})$ bei ungünstigen Pivots.\par
  \item Speicher $O(\log n)$ Rekursionstiefe, instabil, in‑place.\par
  \item Meist schnellstes praktisches Vergleichsverfahren.
\end{itemize}

% =========================================================================
% 3. HEAP VERWENDEN
% =========================================================================
\section{Heap verwenden}
\subsection*{Definition}
Ein \emph{Heap} ist ein vollständiger Binärbaum, der die Heap‑Eigenschaft erfüllt.

\subsection*{Kernoperationen (Pseudocode)}
\begin{algorithmic}
\Function{insert}{int x}
  \State $heap[++size] \gets x$; \Call{shiftUp}{size}
\EndFunction
\Function{extractMax}{}
  \State $max \gets heap[1]$; $heap[1] \gets heap[size--]$;
  \State \Call{shiftDown}{1}; \Return $max$;
\EndFunction
\end{algorithmic}

\subsubsection*{Beispiel – Max‑Heap Aufbau}
Eingabe \([4,\,3,\,1,\,5,\,2]\);
\begin{center}
\begin{tabular}{c|c}
Einfügen & Heap‑Array (1‑basiert)\\\hline
4 & 4\\
3 & 4\,3\\
1 & 4\,3\,1\\
5 & 5\,4\,1\,3\\
2 & 5\,4\,2\,3\,1
\end{tabular}
\end{center}

% =========================================================================
% 4. HEAP SORT
% =========================================================================
\section{Heap‑Sort anwenden}
\subsection*{Idee}
\textbf{Build‑Heap} und dann wiederholt Maximum nach hinten tauschen.

\subsubsection*{Beispiel}
Start‑Heap aus vorherigem Beispiel. Entferne iterativ Maximum:
\[
5\,\underline{4\,2\,3\,1}\;\rightarrow\;1\,\underline{4\,2\,3}\;\rightarrow\;1\,2\,\underline{3}\;\rightarrow\;1\,2\,3\,4\,5
\]

\subsection*{Eigenschaften}
Laufzeit garantiert $O(n\log n)$, Speicher $O(1)$, instabil.

% =========================================================================
% 5. MERGE SORT
% =========================================================================
\section{Merge‑Sort anwenden}
\subsection*{Idee}
Rekursiv halbieren, dann zusammenführen.

\subsubsection*{Beispiel (Top‑Down)}
\begin{align*}
[4,3,1,5,2] &\Rightarrow [4,3] \,[1,5,2]\\
&\Rightarrow [4] [3] \;[1] [5,2]\\
&\Rightarrow [2,5] (Merge)\\
&\Rightarrow [3,4] (Merge)\\
&\Rightarrow [1,2,5] (Merge)\\
&\Rightarrow [1,2,3,4,5] (Final‑Merge)
\end{align*}

\subsection*{Eigenschaften}
Immer $O(n\log n)$, Speicher $O(n)$, stabil.

% =========================================================================
% 6. COUNTING SORT
% =========================================================================
\section{Counting‑Sort anwenden}
\subsection*{Beispiel}
Eingabe \([4,3,1,5,2] ,\;k=6\)
\[
C=\begin{array}{c|cccccc}
val&0&1&2&3&4&5\\\hline
count&0&1&1&1&1&1\end{array}
\;
\Rightarrow\;\text{Prefix}=[0,1,2,3,4,5]\;
\Rightarrow\;\text{Output}=[1,2,3,4,5]
\]

\subsection*{Komplexität}
$O(n+k)$ Zeit/Speicher, stabil.

% =========================================================================
% 7. BUCKET SORT
% =========================================================================
\section{Bucket‑Sort anwenden}
\subsection*{Beispiel}
Normiere Array \([0.42,0.32,0.23,0.52,0.12]\) mit $m=5$ Buckets.
\begin{center}
\begin{tabular}{c|l}
Bucket & Elemente nach Streuung / Sortierung\\\hline
0 & 0.12\\
1 & 0.23\,0.32\\
2 & 0.42\\
3 & 0.52\\
4 & –
\end{tabular}
\end{center}
Konkatenation ergibt \([0.12,0.23,0.32,0.42,0.52]\).

\subsection*{Komplexität}
Average $O(n)$, Worst $O(n^{2})$.

% =========================================================================
% 8. RADIX SORT
% =========================================================================
\section{Radix‑Sort anwenden}
\subsection*{Beispiel (Dezimal, LSD, $d=2$)}
Eingabe \([23,17,45,11,29]\)
\begin{enumerate}[label=Stelle~\arabic*]
  \item Einer‑Sortierung → \([11,23,45,17,29]\)
  \item Zehner‑Sortierung → \([11,17,23,29,45]\)
\end{enumerate}

\subsection*{Komplexität}
$O(d(n+b))$ Zeit.
