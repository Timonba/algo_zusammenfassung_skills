% chapters/algdatgrundlagen.tex – Algorithmen & Datenstrukturen – Grundlagen
% Benötigt im Hauptdokument keine zusätzlichen Pakete neben main.tex

% -----------------------------------------------------------------------------
\chapter{Algorithmen \& Datenstrukturen – Grundlagen}

% ##############################################################################
% 1. VERKETTETE LISTEN
% ##############################################################################
\section{Umgang mit verketteten Listen}
\subsection*{Grundidee}
Eine \emph{einfach verkettete Liste} besteht aus Knoten der Form
\[\texttt{\{ key, next \}}\].
Im Gegensatz zu Arrays wächst die Struktur dynamisch, weil Einfügen/Löschen
nur Zeiger (bzw. Referenzen) ändert.

\subsection*{Typische Operationen und Laufzeiten}
\begin{center}
  \renewcommand{\arraystretch}{1.15}
  \begin{tabular}{@{}lcc@{}}
    \textbf{Operation} & \textbf{am Kopf} & \textbf{allg. Position}\\\hline
    \textsc{insert}   & $\Theta(1)$   & $\Theta(1)$\,*\\
    \textsc{delete}   & $\Theta(1)$   & $\Theta(1)$\,*\\
    \textsc{search}   & \multicolumn{2}{c}{$\Theta(n)$}\\
  \end{tabular}
\end{center}
\emph{\small *\,Vorausgesetzt, ein Verweis auf den Vorgängerknoten ist gegeben.}

\subsection*{Beispielimplementierung (Java)}
\begin{verbatim}
static class Node {
    int key;
    Node next;
    Node(int k, Node n) { key = k; next = n; }
}

static Node pushFront(Node head, int x) {
    return new Node(x, head);
}

static Node popFront(Node head) {
    return (head == null) ? null : head.next;
}
\end{verbatim}
Doppelt verkettete Listen ergänzen ein Feld \texttt{prev} und erlauben damit
$\Theta(1)$ Löschungen ohne Vorgängerzugriff.

\vspace{1em}\noindent
\textbf{Praxisfaustregel:} \emph{Listen für häufige Ein-/Ausfügeoperationen,
Arrays für zufälligen Direktzugriff.}

% ##############################################################################
% 2. SIMULATION EINER DS MIT EINER ANDEREN
% ##############################################################################
\section{Simulation einer Datenstruktur mit einer anderen}
\subsection*{Queue mit zwei Stacks}
\begin{algorithmic}[H]
\Require Zwei Stacks $S_{\text{in}},\,S_{\text{out}}$
\Function{enqueue}{$x$} \Comment{immer $\Theta(1)$}
    push $x$ auf $S_{\text{in}}$
\EndFunction
\Function{dequeue}{} \Comment{amortisiert $\Theta(1)$}
    \If{$S_{\text{out}}$ leer}
        \While{$S_{\text{in}}$ nicht leer} \Comment{jedes Element max. 1\,$\times$ verschoben}
            verschiebe oberstes Element nach $S_{\text{out}}$
        \EndWhile
    \EndIf
    return pop von $S_{\text{out}}$
\EndFunction
\end{algorithmic}

\subsection*{Stack mit zwei Queues}
\textbf{Lazy‑Variante} (\textsc{push} billig, \textsc{pop} teuer):
\begin{algorithmic}[H]
\Require Zwei Queues $Q_1,\,Q_2$
\Function{push}{$x$}
    enqueue $x$ in $Q_1$  \Comment{$\Theta(1)$}
\EndFunction
\Function{pop}{}
    \While{$|Q_1| > 1$} verschiebe Front von $Q_1$ nach $Q_2$ \EndWhile
    swap $Q_1, Q_2$; return letztes Element \Comment{$\Theta(n)$}
\EndFunction
\end{algorithmic}
\textbf{Eager‑Variante} dreht das Kostenprofil um.

\subsection*{Cursor‑Listen (Array \texorpdfstring{$\leftrightarrow$}{↔} Liste)}
Ein Array simuliert Zeiger durch Indizes – nützlich, wenn echte Zeiger verboten
oder Speicher abstrahiert wird (z.\,B. Prüfungen).

% ##############################################################################
% 3. BAUMPARAMETER
% ##############################################################################
\section{Baumparameter berechnen}
\subsection*{Definitionen}
\begin{description}
  \item[Größe] $|B|$ … Anzahl Knoten.
  \item[Blätter] $\mathrm{leaf}(B)$ … Knoten Grad $0$.
  \item[Höhe] $h(B)$ … Länge des längsten Pfades Wurzel $\rightarrow$ Blatt.
\end{description}

\subsection*{Rekursive Berechnung (Pseudocode)}
\begin{verbatim}
size(v):
  if v == null return 0
  return 1 + size(v.left) + size(v.right)

treeHeight(v):                leafCount(v):
  if v == null return -1        if v == null return 0
  return 1 + max(               if v.left==null && v.right==null return 1
      treeHeight(v.left),       return leafCount(v.left)
      treeHeight(v.right))                + leafCount(v.right)
\end{verbatim}
Alle Algorithmen traversieren jeden Knoten einmal: $\Theta(n)$ Zeit und
$\Theta(h)$ Zusatzspeicher durch den Rekursionsstapel.

\subsection*{Grenzen für die Höhe}
Für max. Kind‐Grad $d$ und $n$ Knoten gilt
\[\boxed{\;\lceil\log_d n\rceil \;\le\; h(B) \;\le\; n-1\;}\]

% ##############################################################################
% 4. BAUMDURCHLÄUFE
% ##############################################################################
\section{Baumdurchläufe (inkl. Level‑order)}
\begin{enumerate}
  \item \textbf{Preorder}: Wurzel $\rightarrow$ links $\rightarrow$ rechts
  \item \textbf{Inorder}: links $\rightarrow$ Wurzel $\rightarrow$ rechts (BST $\rightarrow$ sortiert)
  \item \textbf{Postorder}: links $\rightarrow$ rechts $\rightarrow$ Wurzel
  \item \textbf{Level‑order} (BFS): Ebenenweise links $\rightarrow$ rechts
\end{enumerate}

\subsection*{Iterative Level‑order in Java}
\begin{verbatim}
static void levelOrder(Node root) {
    Queue<Node> q = new ArrayDeque<>();
    q.add(root);
    while (!q.isEmpty()) {
        Node v = q.remove();
        visit(v);
        if (v.left  != null) q.add(v.left);
        if (v.right != null) q.add(v.right);
    }
}
\end{verbatim}
Laufzeit $\Theta(n)$, Speicher $\Theta(w)$ mit $w$ = maximale Breite.

% ##############################################################################
% 5. ALGORITHMENANSÄTZE ERKENNEN
% ##############################################################################
\section{Ansatz eines Algorithmus erkennen}
\begin{description}
  \item[Divide \& Conquer] rekursive Zerlegung (z.\,B. Mergesort).
  \item[Backtracking] Tiefensuche mit Zurückrudern (z.\,B. 8‑Damen‑Problem).
  \item[Dynamische Programmierung] überlappende Teilprobleme, Tabellierung.
  \item[Greedy] Wahl lokal optimaler Schritte ohne spätere Korrektur
        (z.\,B. Kruskal‑MST).
\end{description}
Charakteristisch sind typische Code‑Gerüste: Rekursion + Merge, Tabellenschleife,
fortlaufender "beste Wahl"‑Schritt…

% ##############################################################################
% 6. ALGORITHMUS ENTWICKELN
% ##############################################################################
\section{Algorithmus aus Vorgaben entwickeln}
\begin{enumerate}
  \item \textbf{Problem analysieren}: Eingabe/Output exakt definieren.
  \item \textbf{Strategie wählen}: D\&C, DP, Greedy, Backtracking…
  \item \textbf{Datenstrukturen}: Array, Liste, Heap, Graph…
  \item \textbf{Pseudocode} \& Invarianten festhalten.
  \item \textbf{Komplexität}: Landau, evtl. Master‑Theorem für Rekursionen.
\end{enumerate}

% ##############################################################################
% 7. UNPASSENDEN ANSATZ BEGRÜNDEN
% ##############################################################################
\section{Unpassenden Ansatz begründen / Gegenbeispiel finden}
\subsection*{Vorgehen}
\begin{enumerate}
  \item Kernannahme des Ansatzes identifizieren.
  \item Eingabe konstruieren, die Annahme widerlegt.
  \item Falsche Ausgabe oder höhere Kosten nachweisen.
\end{enumerate}
\subsection*{Beispiel: Greedy beim Rucksackproblem}
Greedy wählt stets maximales \emph{Wert/Gewicht}.  Gegenbeispiel:
\[
  (w,v)=\{(5,9),(6,10),(1,1)\},\;C=7
\]
Greedy nimmt $(5,9)+(1,1)$ (Wert 10), optimal ist $(6,10)$.

