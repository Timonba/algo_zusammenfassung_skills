% chapters/laufzeit.tex – Laufzeit von Algorithmen
% (benötigt keine zusätzlichen Pakete außer denen aus main.tex)

\chapter{Laufzeit von Algorithmen}

%--------------------------------------------------------
\section{Definition von $O,\Omega,\Theta$}
\subsection*{Mathematische Formulierung}
Seien $f,g\colon \mathbb N \to \mathbb R_{>0}$ Funktionen. Dann definiert die
\emph{Landau-Notation} die Mengen
\begin{align*}
  O(f)      &:= \{\,g \mid \exists\,c,n_0>0: \forall n\ge n_0:\; g(n)\le c\,f(n)\,\},\\
  \Omega(f) &:= \{\,g \mid \exists\,c,n_0>0: \forall n\ge n_0:\; g(n)\ge c\,f(n)\,\},\\
  \Theta(f) &:= O(f)\cap\Omega(f).
\end{align*}
Kurz: $g\in O(f)$ liefert eine \emph{obere Schranke}, $g\in\Omega(f)$ eine
\emph{untere Schranke}, und $g\in\Theta(f)$ heißt, daß $f$ und $g$ asymptotisch
gleiche Wachstumsordnung besitzen.

\subsection*{Umgangssprachliche Faustformel}
\begin{description}
  \item[$O$ („O-groß“)] gibt an, wie schnell \emph{spätestens} etwas wächst – eine Garantie nach oben.
  \item[$\Omega$ („Omega-groß“)] gibt an, wie schnell \emph{mindestens} etwas wächst – eine Garantie nach unten.
  \item[$\Theta$ („Theta-groß“)] bedeutet „genau in dieser Größenordnung“ – die Funktion wächst weder wesentlich schneller noch wesentlich langsamer.
\end{description}

%--------------------------------------------------------
\section{Funktionen asymptotisch vergleichen}
Um zwei Funktionen $f$ und $g$ zu vergleichen, prüft man oft den Grenzwert
\[\lim_{n\to\infty} \frac{f(n)}{g(n)}\;=\;\begin{cases}
  0 &\Rightarrow f\in O(g)\\[4pt]
  \infty &\Rightarrow f\in \Omega(g)\\[4pt]
  c\in(0,\infty) &\Rightarrow f\in \Theta(g).
\end{cases}\]
Typische Wachstumshierarchie:
\[1\;\ll\; \log\!n\;\ll\; n^k\;\ll\; n^k\log n\;\ll\; a^n\;\ll\; n!\;\ll\; n^{n}.\]
Ein schneller Test ist daher, Polynom- gegen Exponential- oder Fakultäts-Term
zu vergleichen; stetig vorliegende $\log$-Faktoren spielen dabei nur eine
untergeordnete Rolle.

%--------------------------------------------------------
\section{Master-Theorem}
Betrachte Rekursionen der Form
\[T(n)=a\,T\!\bigl(\tfrac{n}{b}\bigr)+d(n)\qquad(a\ge1,\;b>1).\]
Sei $d(n)\in O(n^{\gamma})$ für ein $\gamma\ge0$. Dann gilt
\[
T(n)=\begin{cases}
  \Theta\bigl(n^{\log_b a}\bigr), & a> b^{\gamma},\\[4pt]
  \Theta\bigl(n^{\gamma}\log n\bigr), & a= b^{\gamma},\\[4pt]
  \Theta\bigl(n^{\gamma}\bigr), & a< b^{\gamma}.
\end{cases}
\]
\subsection*{Beispiele}
\begin{enumerate}
  \item[Fall 1:] $T(n)=2T(n/2)+1$\\
        $a=2,\;b=2,\;d(n)=1\in O(n^{0})$ mit $a>b^{0}=1$ $\Rightarrow$
        $T(n)=\Theta(n^{\log_22})=\Theta(n)$.\\
        (Lineare Zeit, etwa beim Best-case von Quicksort.)
  \item[Fall 2:] $T(n)=2T(n/2)+n$\\
        $a=2,\;b=2,\;d(n)=n$ mit $\gamma=1$ und $a=b^{\gamma}$ $\Rightarrow$
        $T(n)=\Theta(n\log n)$.\\
        (Klassisch: Mergesort.)
  \item[Fall 3:] $T(n)=9T(n/3)+n$\\
        $a=9,\;b=3,\;d(n)=n$ mit $\gamma=1$ und $a>b^{\gamma}=3$ $\Rightarrow$
        $T(n)=\Theta\bigl(n^{\log_39}\bigr)=\Theta(n^{2})$.\\
        (Beispiel aus Strassen-Matrixmultiplikation.)
\end{enumerate}

%--------------------------------------------------------
\section{Erzeugende Funktionen}
Gegeben eine Folge $(a_n)_{n\ge0}$ definiert die
\emph{erzeugende Funktion}
\[F(x)=\sum_{n=0}^{\infty} a_n x^{n}.\]
Vorteil: Verschiebungen werden zu einfachen Operationen, z.\,B.\
$x\,F(x)=\sum_{n\ge0} a_{n-1}x^{n}$ (Index-Shift).

\subsection*{Idee}
Eine Rekursion für $a_n$ wird in eine Gleichung für $F(x)$ übersetzt, diese nach
$F(x)$ aufgelöst und anschließend über Potenzreihenentwicklung $a_n$
ausgelesen.

\subsection*{Beispiel: Geometrische Folge}
Sei $a_n=2^n$ mit $a_0=1$. Dann
\[F(x)=\sum_{n\ge0} (2x)^n=\frac{1}{1-2x},\qquad |x|<\tfrac{1}{2}.\]
Die geschlossene Form verrät Radius der Konvergenz $\tfrac12$ und bestätigt
$a_n=[x^{n}]F(x)=2^{n}$.

\subsection*{Beispiel: Fibonacci}
Für $F(x)=\sum_{n\ge0} \text{Fib}(n) x^n$ gilt wegen $\text{Fib}(n)=\text{Fib}(n-1)+\text{Fib}(n-2)$
\[F(x)=x + xF(x) + x^{2}F(x) \;\Longrightarrow\; F(x)=\frac{x}{1 - x - x^{2}}.\]
Eine Partialbruchzerlegung liefert schließlich die Binet-Formel
\[\text{Fib}(n)=\frac{\varphi^{n}-\psi^{n}}{\sqrt{5}},\qquad
  \varphi=\tfrac{1+\sqrt{5}}{2},\;
  \psi=1-\varphi.\]
